\newpage
\vspace*{4cm}
\begin{center}
\textbf{CONCLUSÃO}
\end{center}
\vspace{48pt}

\addcontentsline{toc}{section}{CONCLUSÃO}

\hspace*{0.8cm}O objetivo deste trabalho foi desenvolver um  protótipo de aquisição de sinais eletrocardiográficos portátil de baixo custo, para obter as ondas P, complexos QRS e ondas T. Inicialmente, realizou-se uma revisão das áreas de conhecimento relativo aos sinais que constituem o eletrocardiograma, estrutura de aquisição e transmissão de dados, tendo em vista atender a proposição do trabalho de pesquisa.

O protótipo desenvolvido, conseguiu atender as especificações iniciais, gerando resultados satisfatórios que permitem a análise do sinal e a comunicação com um computador em tempo real, através apenas de uma solicitação utilizando uma rede de acesso sem fio, empregando o protocolo 802.11.

Com base no protótipo desenvolvido, é possível analisar diretamente a forma de onda P, complexo QRS e onda T, e verificar a influência do batimento cardíaco sobre o sinal. Foram realizados testes com a frequência mínima de 30 btm e frequência máxima de 240 btm, para assim,  determinar a oxigenação do corpo e possíveis disfunções à saúde, ou até mesmo casos críticos que podem levar a morte repentina. \textcolor{red}{ESPAÇOS NOVAMENTE}

Ao decorrer do trabalho surgiram novas implementações que  foram desenvolvidas, porém sem êxito, sendo uma destas a implementação do cabo paciente. O protótipo foi desenvolvido para receber a conexão do cabo paciente (AE-0H002-0) , porém os resultados da forma de onda ECG, não foram satisfatórios \textcolor{red}{TU MOSTROU ISSO NO CAP DE RESULTADOS??? SE NÃO, COLOQUE}. Outras ideias para melhoria do sistema não foram desenvolvidas, devido ao espaço de tempo limitado e a complexidade envolvida 

Com base nisso merecem destaque os seguintes trabalhos futuros.

\begin{itemize}
\item Trabalho 1
\item Trabalho 2
\item Trabalho 3
\end{itemize}

É importante mencionar que as métricas utilizadas para adaptar a eficiência dos grupos geradores operando com Biodiesel foram estimadas com bases em trabalhos publicados acerca desse tópico. Logo, um trabalho específico para aferir de forma prática é de grande importância para a continuidade da pesquisa.