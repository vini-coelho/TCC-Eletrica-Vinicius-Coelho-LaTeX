\newpage
%\vspace*{4cm}
\setcounter{page}{12}
\begin{center}
\section{INTRODUÇÃO}
%\textbf{INTRODUÇÃO\\\\}
\end{center}
\par
%\addcontentsline{toc}{section}{INTRODUÇ\~AO}	

	\hspace{0.8cm}Devido à forte crise do setor elétrico dos últimos anos, os consumidores brasileiros de energia elétrica têm convivido com sucessivos aumentos tarifários e sofrido com inúmeras falhas no fornecimento, principalmente, durante os horários de maior consumo \cite{kurek}. Agora, os consumidores, particularmente, do setor produtivo, passam a ver as ameaças de racionamento e black-outs de energia com sério risco de se tornarem realidade. 
    
    Com essa crescente perda de confiabilidade do sistema de fornecimento de energia, muitas empresas têm utilizado grupos geradores como fonte backup de energia, a fim de evitar paradas e garantir a continuidade dos processos produtivos, mesmo em meio aos incidentes de falta de energia. Nesse cenário, a demanda por geradores movidos a diesel tem crescido significativamente e fabricantes relatam alta de até 35\% nos pedidos e cotações por geradores a diesel e a gás \cite{fucuchima}.
    
    Entretanto, limitar o uso desses equipamentos somente a este fim, implica numa subutilização de recurso, que leva ao questionamento sobre a validade do investimento. 
   
    Com o aumento das tarifas, especialmente nos horários de ponta, o uso de grupos geradores se apresenta como uma alternativa plausível para atuar na cobertura de picos de carga e, assim, evitar o custo por incidência das altas penalidades (sob tarifas ainda mais caras) aplicadas pelas concessionárias àqueles que violam os limites estabelecidos em contrato. Mesmo em horários de baixo consumo, os grupos geradores podem oferecer alternativa de menor custo na geração de energia elétrica, principalmente, se abastecidos com combustíveis mais baratos.
    
    Este trabalho apresenta um estudo de caso das instalações do Tribunal de Contas do Estado do Amazonas (TCE -- AM), com o objetivo de realizar um levantamento do atual estado das instalações e recursos de geração de energia, visando descrever um método para implementar a utillização dos grupos geradores já presentes na unidade de forma ativa, para geração de energia na ponta, utilizando o Biodiesel como biocombustível, aferindo a viabilidade econômica e ambiental desse esforço.
    
\subsection{TEMA}
\subsubsection{Delimitação do tema}

\subsection{PROBLEMA DE PESQUISA}

    \hspace{0.8cm}Num cenário de crescente crise no sistema energético brasileiro, empresas de diversos setores tem manifestado interesse na busca de soluções alternativas que possibilitem a redução dos gastos com energia elétrica. Nesse contexto, muitas empresas utilizam grupos geradores como fonte backup na prevenção de paralizações por falta de energia. Entretanto, limitar o uso desses equipamentos somente a este fim, implica numa subutilização de recurso, que leva ao questionamento sobre a validade do investimento. 
    
    Com o aumento das tarifas, especialmente nos horários de ponta, o uso de grupos geradores se apresenta como uma alternativa plausível para atuar na cobertura de picos de carga e, assim, evitar o custo por incidência das altas penalidades (sob tarifas ainda mais caras) aplicadas pelas concessionárias àqueles que violam os limites estabelecidos em contrato. Mesmo em horários de baixo consumo, os grupos geradores podem oferecer alternativa de menor custo na geração de energia elétrica, principalmente, se abastecidos com combustíveis mais baratos.
    
    No entanto, o sucesso na implementação dessa estratégia depende diretamente do custo em um investimento contínuo: o combustível diesel. Com a alta nos preços do petróleo, o preço desse combustível vem aumentando numa escala superior ao da inflação nos últimos anos. De acordo com a pesquisa da Vale Card, uma empresa especializada em soluções de gestão de frotas, o preço médio do diesel comum subiu 25,26\% neste período. Já a inflação teve uma variação de 18,42\%. \cite{janone}. 
    
    Nesse contexto, a utilização do Biodiesel, uma fonte renovável de energia, pode ser uma alternativa para alimentação dos grupos geradores não só de menor custo como menos poluente e de baixo impacto ao meio ambiente. Este trabalho propõe a utillização do biodiesel como combustível dos grupos geradores, aplicando-os de forma ativa para redução de custos de energia elétrica.


\subsection{OBJETIVOS}
\subsubsection{Objetivo geral}
\subsubsection{Objetivos específicos}

\subsection{JUSTIFICATIVA}

A pesquisa propõe uma alternativa de utilização dos grupos geradores, primariamente utilizados como fonte reserva de energia, para trazer economias nas tarifas pagas e redução de custos, frente ao cenário de crise energética e aumento significativo nas tarifas de consumo de energia elétrica, tendo como objeto de estudo as instalações do Tribunal de Contas do Estado do Amazonas.

Para isso, propõe-se a utilização do biodiesel como combustível dessas máquinas geradoras, que é uma fonte de energia renovável e de menor custo em comparação ao diesel convencional de origem fóssil. Além de baratear a utilização dos grupos geradores, a proposta tem grande potencial de redução de emissão de poluentes. Segundo a \citeonline{embrapa}, uma mistura de 20\% de biodiesel em diesel fóssil reduz em 70\% a emissão de gases do efeito estufa.


\subsection{METODOLOGIA}

\subsection{ESTRUTURA DO TRABALHO}

Este trabalho foi dividido nos seguintes capítulos:

\begin{itemize}
    \item    Capíutlo 1: Introdução do trabalho, abordando o tema e sua delimitação, o problema de pesquisa abordado, os objetivos da pesquisam justificativas e metodologia proposta.
    \item Capíutlo 2: 
    \item Capíutlo 3
    \item Capíutlo 4
    \item Capíutlo 5
\end{itemize}

   
    
    
